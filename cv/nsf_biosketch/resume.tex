\documentclass[10pt]{article}

%\usepackage{latex8}
\usepackage{url}
\usepackage{titlesec}

\titleformat*{\section}{\large\bfseries}

%\usepackage{helvet} \renewcommand{\familydefault}{\sfdefault} 

\usepackage[ManyBibs,NoDate]{currvita}
\usepackage{fullpage}
\usepackage{bibunits}
%\usepackage{url}

%\textwidth 6.0in
%\oddsidemargin -0.5in
%\evensidemargin -0.5in
%\textheight 8in
\topmargin 0.5in

\newcounter{bibctr}

\newcommand{\mkbib}[1]{
    \refstepcounter{bibctr}
    \item[\hskip0pt plus1filll \hbox{[\arabic{bibctr}]}]
    \begin{bibunit}[abbrv]
      \nocite{#1}
      \putbib[self]
      \label{bib:#1}
    % A bit more space
    \vspace{5pt}
    \end{bibunit}
}

\newcommand{\refbib}[1]{[\ref{bib:#1}]}

\newcommand{\ie}{{\em i.e.}}
\newcommand{\eg}{{\em e.g.}}

%\def\up#1{\raise.16ex\hbox{#1}}

\pagestyle{empty}

%\title{Bill Howe}

\begin{document}
\begin{cv}{}

%\maketitle
%\thispagestyle{empty}
\centerline{\Large Bill Howe}
\vspace{0.5cm}

\noindent University of Washington, Department of Computer Science and Engineering \\
\noindent Box 351202, Seattle, WA 98195-2350 

\section*{Professional Preparation}
\noindent
\begin{tabular}{p{0.30\linewidth}p{.15\linewidth}p{0.45\linewidth}p{0.10\linewidth}}
Georgia Institute of Technology &  Atlanta, GA & BS, Honors, Industrial and Systems Engineering & 1999 \\
Portland State University & Portland, OR & PhD, Commendation, Computer Science & 2006 \\
\end{tabular}

\section*{Appointments}
\noindent
\begin{tabular}{p{0.15\linewidth}p{0.85\linewidth}}

2014-present & Associate Director, eScience Institute, University of Washington \\
2014-present & Affiliate Associate Professor, Computer Science and Engineering, University of Washington \\
2012-2014 & Director of Research, Scalable Data Analytics, eScience Institute, University of Washington \\
2009-2014 & Affiliate Assistant Professor, Computer Science and Engineering, University of Washington \\
2009-2012 & Senior Scientist, eScience Institute, University of Washington \\
2008-2009 & Staff Scientist, NSF Science and Technology Center for Coastal Margin Observation and  Prediction, Oregon Health \& Science University \\
2006-2008 & Senior Research Associate, NSF Science and Technology Center for Coastal Margin Observation and  Prediction, Oregon Health \& Science University \\
2001-2006 & Graduate Research Assistant, Portland State University \\
1999-2001 & Consultant, Deloitte Consulting, Microsoft, Schlumberger Inc., Siebel Systems. \\
\end{tabular}

\section*{Awards and Honors}
{
\renewcommand\arraystretch{1.3}% (MyValue=1.0 is for standard spacing)

 \begin{tabular}{p{\linewidth}}
\textbf{Best of VLDB 2010}, selected for publication in VLDB Journal special issue, for ``Haloop: Efficient iterative data processing on large clusters.'' (currently the most cited paper from VLDB 2010)\\
\textbf{Jim Gray Seed Award}, Microsoft Research, \$40,000, April 2010.  Project: Client+Cloud: Bridging  the Gap Between Spreadsheets and Databases for eScience  \\
\textbf{Jim Gray Seed Award}, Microsoft Research, \$25,000, April 2008.  Project: The eScience Appliance \\
\textbf{Department Commendation Award}, Maseeh College of Engineering and Computer Science, Portland State University, May 2007 \\
\textbf{Best of VLDB 2004}, one of 4 best papers (of 81) selected for publication in VLDB Journal special issue, for ``Algebraic manipulation of scientific datasets''\\
 \end{tabular}
}

\vspace{-0.7cm}
\section*{Publications most closely related to the proposal}

\vspace{-0.7cm}
\begin{cvlist}{{}}
\item 
\mkbib{ren:13} 
\mkbib{shaw:12} 
\mkbib{kwon:12} 
\mkbib{vo:11} % LDAV/ Hadoop
\mkbib{bu:10}       % HaLoop
%\mkbib{howe:jei:03}      
\end{cvlist}

\section*{Other publications}

\vspace{-0.6cm}
\begin{cvlist}{{}}
\item 
%\mkbib{howe:ultravis:09}

\mkbib{howe:11}
\mkbib{kwon:10}
\mkbib{grochow:10} 
\mkbib{howe:ssdbm:09}       % Scientific applications
\mkbib{howe:cidr:07}        % Scientific data management requirements
%\mkbib{howe:iimas:08} % Scientific data management
%\mkbib{howe:escience:08}    % Distrbuted systems; scientific data analysis
%\mkbib{baptista:cse:08} % Scientific data systems   
%\mkbib{howe:vldbjournal:04}  % Query processing on simulation data
%\mkbib{hhowe:jei:03}         % Scientific applications
\end{cvlist}

\section*{Synergistic Activities}


%A list of up to five examples that demonstrate the broader impact of the 
%individual\u2019s professional and scholarly activities that focuses
%on the integration and transfer of knowledge as well as its
%creation. Examples could include, among others: innovations in
%teaching and training (e.g., development of curricular materials and
%pedagogical methods); contributions to the science of learning;
%development and/or refinement of research tools; computation
%methodologies, and algorithms for problem-solving; development of
%databases to support research and education; broadening the
%participation of groups underrepresented in science, mathematics,
%engineering and technology; and service to the scientific and
%engineering community outside of the individual\u2019s immediate
%organization.

\noindent {\bf eScience Institute}. Director of Research, Scalable Data Analytics; outreach and consulting for big data in the sciences (http://escience.washington.edu)

\noindent {\bf Course development}. Designed a coursera MOOC ``Introduction to Data Science'' with over 100,000 registrants and 9,000 earned certificates Spring 2013; Developed a new Introductory "Data Programming" Course, Summer 2012; New course ``Data-Intensive Computing in the Cloud,'' Spring 2012; Advisory Board, Data Science Certificate, UW Educational Outreach; Advisory Board, Cloud Computing Certificate, UW Educational Outreach; ``Scientific Data Management'' (2010), University of Washington (with Magdalena Balazinska).

\noindent {\bf Science Advisory Board Member}. SciDB project, http://scidb.org/

\noindent{\bf Organizing Committee} Co-Chair, DataMASS 2012; Demo Co-chair, SSDBM 2013; Chair, HPCDB 2011/2012; XLDB Organizing Committee, 2011; Co-Chair, Workshop on Array Databases, 2011 (with Peter Baumann); Registration Chair, SSDBM 2011.

\noindent{\bf Reviewer}.   Reviewer, PVLDB, 2012-2013; Program Committee, LDAV 2013; Program Committee, ScienceCloud 2012; Reviewer, VLDB Journal, 2011; Program Committee, EDBT 2011; Demo Program Committee, SIGMOD 2011; Registration Chair, SSDBM 2011. Program Committee Program Committee, SSDBM 2010. Reviewer, Journal of Parallel and Distributed Computing, May 2010.  Reviewer, VLDB Journal, 2007

\section*{Collaborators and other affiliations}

    %* 48 MONTHS - Collaborators and Co-Editors. A list of all persons in
    %* alphabetical order (including their current organizational
    %* affiliations) who are currently, or who have been collaborators
    %* or co-authors with the individual on a project, book, article,
    %* report, abstract or paper during the 48 months preceding the
    %* submission of the proposal. 

    %  24 MONTHS - Also include those individuals who
    %* are currently or have been co-editors of a journal, compendium,
    %* or conference proceedings during the 24 months preceding the
    %* submission of the proposal. If there are no collaborators or
    %* co-editors to report, this should be so indicated.

    %* Graduate Advisors and Postdoctoral Sponsors. A list of the names
    %* of the individual\u2019s own graduate advisor(s) and principal
    %* postdoctoral sponsor(s), and their current organizational
    %* affiliations.

    %* Thesis Advisor and Postgraduate-Scholar Sponsor. A list of all
    %* persons (including their organizational affiliations), with whom
    %* the individual has had an association as thesis advisor, or with
    %* whom the individual has had an association within the last five
    %* years as a postgraduate-scholar sponsor. The total number of
    %* graduate students advised and postdoctoral scholars sponsored
    %* also must be identified.

\noindent {\bf Collaborators (outside the University of Washington)}:

 \begin{tabular}{p{\linewidth}}
Matthew Arrott (UCSD),
Roger Barga (Microsoft Research), 
Mike Cafarella (University of Michigan),
Ian Foster (University of Chicago),
Christine Borgman (UCLA),
Bryan Heidorn (Arizona),
Carl Kesselman (USC),
Lois Delcambre (Portland State University),
Juliana Freire (NYU Poly),
David Maier (Portland State University),
Jerome Rolia (HP),
Rich Signell (USGS),
Claudio Silva (NYU Poly)
\end{tabular}

\noindent {\bf Advisors}:  David Maier, Portland State University, Thesis Advisor; Antonio Baptista, Oregon Health \& Science University, Postdoctoral Advisor

\end{cv}

\end{document}

