\documentclass[10pt]{article}

%\usepackage{latex8}
\usepackage{url}
\usepackage{titlesec}

%\titleformat*{\section}{\large\bfseries}

%\usepackage{helvet} \renewcommand{\familydefault}{\sfdefault} 

\usepackage[ManyBibs,NoDate]{currvita}
\usepackage{fullpage}
\usepackage{bibunits}
%\usepackage{url}


%\textwidth 6.0in
%\oddsidemargin -0.5in
%\evensidemargin -0.5in
%\textheight 8in
\topmargin 0.5in

\newcounter{bibctr}

\newcommand{\mkbib}[1]{
    \refstepcounter{bibctr}
    \item[\hskip0pt plus1filll \hbox{[\arabic{bibctr}]}]
    \begin{bibunit}[abbrv]
      \nocite{#1}
      \putbib[self]
      \label{bib:#1}
    % A bit more space
    \vspace{5pt}
    \end{bibunit}
}

\newcommand{\refbib}[1]{[\ref{bib:#1}]}

\newcommand{\ie}{{\em i.e.}}
\newcommand{\eg}{{\em e.g.}}

%\def\up#1{\raise.16ex\hbox{#1}}

\pagestyle{empty}

%\title{Bill Howe}

\begin{document}
\begin{cv}{}

%\maketitle
%\thispagestyle{empty}
\centerline{\Large Bill Howe}
\vspace{0.5cm}

\noindent University of Washington, Department of Computer Science and Engineering \\
\noindent Box 351202, Seattle, WA 98195-2350 

\section*{Professional Preparation}
\noindent
\begin{tabular}{p{0.30\linewidth}p{.15\linewidth}p{0.45\linewidth}p{0.10\linewidth}}
Georgia Institute of Technology &  Atlanta, GA & BS, Honors, Industrial and Systems Engineering & 1999 \\
Portland State University & Portland, OR & PhD, Commendation, Computer Science & 2006 \\
\end{tabular}

\section*{Appointments}
\noindent
\begin{tabular}{p{0.15\linewidth}p{0.85\linewidth}}

2016-present & Associate Professor, Information School, University of Washington \\
2016-present & Adjunct Associate Professor, Allen School of Computer Science \& Engineering, University of Washington \\
2017-present & Adjunct Associate Professor, Electrical Engineering, University of Washington \\
2013-2016 & Founding Associate Director, eScience Institute, University of Washington \\
2012-2016 & Affiliate Faculty, Computer Science and Engineering, University of Washington \\
2009-2013 & Senior Scientist, eScience Institute, University of Washington \\
2006-2009 & Research Scientist, NSF Science and Technology Center for Coastal Margin Observation and  Prediction, Oregon Health \& Science University \\
2001-2006 & Graduate Research Assistant, Portland State University \\
1999-2001 & Consultant, Deloitte Consulting, Microsoft, Schlumberger Inc., Siebel Systems. \\
\end{tabular}

\section*{Products}

\subsection*{Products Most Closely Related to the Proposed Project}

\begin{cvlist}{{}}
\item 
\mkbib{2019-draco}
\mkbib{grechkin2017ezlearn}
\mkbib{jain2016sqlshare}
\mkbib{ren:13} 
\mkbib{bu:10}       % HaLoop
\end{cvlist}

{\small
\subsection*{Other Significant Products}
}
\begin{cvlist}{{}}
\item 
%\mkbib{howe:ultravis:09}

\mkbib{wongsuphasawat2016voyager}
\mkbib{hyrkas2016clustering}
\mkbib{bae2015gossip}
\mkbib{kwon:10}
\mkbib{howe:vldb:04}
%\mkbib{howe:iimas:08} % Scientific data management
%\mkbib{howe:escience:08}    % Distrbuted systems; scientific data analysis
%\mkbib{baptista:cse:08} % Scientific data systems   
%\mkbib{howe:vldbjournal:04}  % Query processing on simulation data
%\mkbib{hhowe:jei:03}         % Scientific applications
\end{cvlist}

\section*{Synergistic Activities}


%A list of up to five examples that demonstrate the broader impact of the 
%individual\u2019s professional and scholarly activities that focuses
%on the integration and transfer of knowledge as well as its
%creation. Examples could include, among others: innovations in
%teaching and training (e.g., development of curricular materials and
%pedagogical methods); contributions to the science of learning;
%development and/or refinement of research tools; computation
%methodologies, and algorithms for problem-solving; development of
%databases to support research and education; broadening the
%participation of groups underrepresented in science, mathematics,
%engineering and technology; and service to the scientific and
%engineering community outside of the individual\u2019s immediate
%organization.

\noindent {\bf Organizational Leadership}. I served as Founding Associate Director of the UW eScience Institute (http://escience.washington.edu), leading hiring, operations, program development, outreach, and consulting in the sciences.  
I also Co-Founded Urban@UW (http://urban.uw.edu) and the Cascadia Urban Analytics Cooperative (https://www.cascadiadata.org/) working to improve collaborative research in the delivery of human services through data-driven decision-making.  I also founded the UW Masters Degree in Data Science, leading curriculum development and organizational design.

\vspace{0.5em}
\noindent {\bf Data Science Curricula Development}. I led a certificate program in data science in 2011, designed and ran a coursera MOOC ``Introduction to Data Science'' with over 200,000 registrants and 20,000 earned certificates from 2013 forward, developed a new Introductory "Data Programming" Course in Summer 2012, and founded the Masters Degree in Data Science.

\vspace{0.5em}
\noindent {\bf Awards and Honors}.  Best Paper, InfoVis 2019; Best of VLDB 2004 and 2010 (selected for special issue); currently most-cited papers from VLDB 2010 and SIGMOD 2012; Two Jim Gray Seed Award from Microsoft Research in 2008 and 2010; Departmental disseration award 2007, Portland State University.

\vspace{0.5em}
\noindent{\bf Professional Service} Area Chair, SIGMOD 2018-2019; Sponsorship Chair, SIGMOD 2020; Program Chair, eScience 2016; Tutorials Co-Chair, ICDE 2018; Industry Co-chair, ICDE 2017; Demo Co-chair, SSDBM 2013; Chair, HPCDB 2011/2012; XLDB Organizing Committee, 2011; Co-Chair, Workshop on Array Databases, 2011 (with Peter Baumann).
I have regularly served on the program committee for PVLDB, SIGMOD, ICDE, and a number of other venues for the last decade.

\vspace{0.5em}
\noindent{\bf National Academy of the Sciences Roundtable on Post-Secondary Data Science Education}.

\end{cv}

\end{document}

